%-------------------------------------------------------------------------------
% rtl66
%-------------------------------------------------------------------------------
%
% \file        rtl66
% \library     Documents
% \author      Chris Ahlstrom
% \date        2024-06-04
% \update      2024-06-04
% \version     $Revision$
% \license     $XPC_GPL_LICENSE$
%
%     This document provides LaTeX documentation for the Rtl66 library.
%
%-------------------------------------------------------------------------------

\section{High-Level Desciption of Rtl66}
\label{sec:rtl_high_level}

   \index{rtl}
   The \textsl{Rtl66} library refactors the \textsl{RtMidi} and
   \textsl{RtAudio} libraries, and then extends them with additional
   support for MIDI and audio operations.
   Some additional helpful modules (e.g. for I/O threads) have also
   been created.

\subsection{Rtl Extras}
\label{subsec:rtl_extras}

   The "extras" are modules in the top-level directories
   \texttt{include/rtl} and \texttt{src/rtl}.
   They provide functionality common to MIDI and audio support.
   They reside in the \texttt{rtl} namespace.

\subsubsection{Rtl Build Macros}
\label{subsubsec:rtl_build_macros}

   The \texttt{rtl\_build\_macros.h} header file contains macro
   definitions for version information, code export settings,
   various features, and for the creation of the supported APIs.
   The latter are somewhat determined by the build platform (e.g.
   \textsl{Windows} versus \textsl{Linux}.

   In the future, some feature macros might be migrated to the
   \texttt{meson.options} file.

\subsubsection{Rtl rterror}
\label{subsubsec:rtl_rterror}

   The \texttt{rterror} class provides an enumeration to indicate the
   category of error (e.g. a driver error versus a warning),
   an alias for an error callback function, and some virtual message-related
   functions.

\subsubsection{Rtl api\_base}
\label{subsubsec:rtl_api_base}

   The \texttt{api\_base} class currently provides only support for
   the handling of errors and warnings.
   It provides the ability for a caller to set a callback function to
   handle errors, including a pointer to error-callback data.

\subsubsection{Rtl rt\_types}
\label{subsubsec:rtl_rt_types}

   This C++ header file simply contains the following lines:

   \begin{verbatim}`
      #include "rtl/midi/rt_midi_types.hpp"
      #include "rtl/audio/rt_audio_types.hpp"
   \end{verbatim}`

\subsubsection{Rtl iothread}
\label{subsubsec:rtl_iothread}

   The \texttt{iothread} class supports starting an
   \texttt{std::thread} that will run a caller-supplied function object
   \index{functor}
   (also known as a \textsl{functor}).
   It also simplies launching and joining a thread.
   For an example of usage, see the \texttt{midi::player} class.

\subsubsection{Rtl Test Helpers}
\label{subsubsec:rtl_test_helpers}

   Provides some free functions in the \texttt{rtl} namespace.
   They provides some simple functions to choose test I/O port numbers,
   show basic help text, and some other basic settings.

\subsection{Rtl MIDI}
\label{subsec:rtl_midi}

   The \textsl{Rtl66} library consists of the core of \textsl{RtMidi} and
   extensions to it. The core is in the \texttt{rtl} namespace and
   the extensions are mostly in the \texttt{midi} namespace.

\subsubsection{Rtl MIDI Core}
\label{subsubsec:rtl_midi_core}

   This section describes the classes and modules derived from the
   \textsl{RtMidi} library.
   Rather than a monolithic module, the classes have been split across
   multiple modules for easier management.
   They reside in the \texttt{rtl} namespace.

\paragraph{Rtl MIDI midi\_api}
\label{paragraph:rtl_midi_midi_api}

   \index{ABC}
   The \texttt{midi\_api} class is an \textsl{abstract base class}
   ("ABC") for all of the various MIDI OS-related class (ALSA, JACK,
   MacOSX Core MIDI, etc.).
   The actual API in use is selected from one of the compiled-in
   APIs, which depends on the build platform.

   The \texttt{rtmidi} class is assigned a pointer to an instantiated
   MIDI API class, and all the MIDI calls are done through this pointer.

   The functionality covered is:

   \begin{itemize}
      \item \textsl{Current API}.
         The selected API is stored for reuse.
      \item \textsl{Port type}.
         Indicates if the port is for input, output,
         duplex (both input and output), and engine.
         Engine is a concept whereby only one MIDI client needs to be
         created, and the ports can all use this common client.
      \item \textsl{Master buss}.
         The \texttt{masterbus}, if created, provides the common MIDI
         client data for the MIDI "engine".
         If not created, operation is much like that of the \textsl{Rtmidi}
         library.
      \item \textsl{Handles}.
         Our handles are void pointers to API data or API MIDI clients (e.g.
         the \texttt{snd\_seq\_t} handle of ALSA).
      \item \textsl{Engine connection/disconnection}.
         If a \texttt{masterbus} was created, a connection is stored as
         a void pointer.
      \item \textsl{Port enumeration}.
         Functions are provided to find all existing MIDI ports, count them,
         and get their names.
         For JACK, a port alias can also be obtained.
      \item \textsl{Port opening/closing}.
      \item \textsl{Sending}.
         Virtual functions for various send operations are provided.
         Non-functional in this base class.
      \item \textsl{Receiving}.
         Virtual functions for polling and event-retrieval
         operations are provided.
         Non-functional in this base class.
      \item \textsl{Input callback}.
         Callers can optionally provide a callback function to handle
         MIDI input.
      \item \textsl{Miscellaneous}.
         Various other features are provided.
         Unnecessary to describe them here.
   \end{itemize}

\paragraph{Rtl MIDI rtmidi}
\label{paragraph:rtl_midi_rtmidi}

   The \texttt{rtmidi} class is an abstract base class with non-virtual access
   functions for the \texttt{midi\_api} functions.
   The functions to open a regular and a "virtual" port are virtual
   just so the default port name can be set correctly.

   The enumeration \texttt{rtmidi::api} reflects the APIs available
   (though some are not yet written); the pure-virtual function
   \texttt{open\_midi\_api()} is written in
   \texttt{rtmidi\_in}, \texttt{rtmidi\_out}, and \texttt{rtmidi\_engine}.

   The actual functions of the selected API are called through the stored
   \texttt{midi\_api} pointer, if it exists, as in the following
   implementation of \texttt{flush\_port()}:

   \begin{verbatim}
      bool
      rtmidi::flush_port ()
      {
         return rt_api_ptr()->flush_port();
      }
   \end{verbatim}

   Also defined are a number of static functions to gather information
   about the supported APIs and to select the API to be used for the
   run of an application.

   Note that \texttt{rtmidi} does not support the \textsl{Seq66/Rtl66}
   concept of a "master bus".

\paragraph{Rtl MIDI rtmidi\_in}
\label{paragraph:rtl_midi_rtmidi_in}

   The \texttt{rtmidi\_in} class extends \texttt{rtmidi} to
   add functions for setting the input buffer size, setting or canceling
   the input callback function, ignoring MIDI data such as SysEx,
   and a function to get \texttt{midi::message} data.

   It also overrides \texttt{open\_midi\_api()} to try to open
   one of the MIDI APIs for input.

\paragraph{Rtl MIDI rtmidi\_out}
\label{paragraph:rtl_midi_rtmidi_out}

   The \texttt{rtmidi\_in} class extends \texttt{rtmidi} to
   add a function to send \texttt{midi::message} data.

   It also overrides \texttt{open\_midi\_api()} to try to open
   one of the MIDI APIs for output.

\paragraph{Rtl MIDI rtmidi\_engine}
\label{paragraph:rtl_midi_rtmidi_engine}

   The \texttt{rtmidi\_in} class extends \texttt{rtmidi} to
   add a pointer to a \texttt{midi::masterbus}.

   It also overrides \texttt{open\_midi\_api()} to try to open
   one of the MIDI APIs as an "engine" (neither input nor output).
   If successful, then the \texttt{midi::masterbus} pointer is
   passed to the currently-selected MIDI API object.

\paragraph{Rtl MIDI Miscellaneous}
\label{paragraph:rtl_midi_miscellaneous}

   This section briefly describes some additional modules.

   \begin{itemize}
      \item \texttt{find\_midi\_api}.
         Provides free functions to find and try to open a MIDI API object.
         Some platforms provide multiple potential APIs and a fall-back
         API in case the caller does not specify one. For example,
         on \textsl{Linux}, JACK is the fallback API if detected, otherwise
         ALSA is the fallback API.
      \item \texttt{midi\_dummy}.
         This API is a non-functional class to be use only when necessary.
      \item \texttt{midi\_queue}.
         Provides a queue of \texttt{midi::message} objects.
      \item \texttt{rtmidi\_in\_data}.
         Includes a \texttt{midi\_queue} for incoming data, plus
         callbacks, optional user-specific data, buffer size, API data, and
         other book-keeping for MIDI input events.
      \item \texttt{rt\_midi\_types}.
         Currently empty.
      \item \texttt{rtmidi\_c}.
         Provides C-function wrappers for \texttt{rtmidi} functionality.
         Also provide are plain C enumerations corresponding to
         \texttt{rtl::rtmidi::api} and
         \texttt{rtl::rterror::kind}.
      \item \textsl{MIDI API directories}.
         Directories provided are:
         \texttt{alsa},
         \texttt{jack},
         \texttt{macosx},
         \texttt{pipewire},
         \texttt{webmidi}, and
         \texttt{winmm}.
         Their contents will be discussed elsewhere.
         Currently, only \texttt{alsa} and \texttt{jack} are fully implemented.
         Also, there are some new APIs not yet represented.
   \end{itemize}

\subsubsection{Rtl MIDI Extensions}
\label{subsubsec:rtl_midi_extensions}

   They reside in the \texttt{midi} namespace.
   Let's get the tricky "buss" stuff out of the way first.

\paragraph{Rtl MIDI Bus}
\label{paragraph:rtl_midi_extensions_bus}

   The \texttt{midi::bus} class is base/mixin class for providing
   \textsl{Seq66} concepts for \texttt{midi::bus\_in} and
   \texttt{midi::bus\_out}.

   It provides the following concepts:

   \begin{itemize}
      \item \textsl{Master Bus}.
         Like \texttt{rtl::rtmidi\_engine}, this class hooks in a
         \texttt{midi::masterbus}.
      \item \textsl{Bus Index}.
         This value is an ordinal number starting at 0.
         In an application, MIDI busses are referred to by this number, not
         by name.
      \item \textsl{Client ID and Display Name}.
         The client number determined by the MIDI engine, if available.
         The display name is assembled by the \texttt{set\_name()} function.
      \item \textsl{Bus ID and Bus Name}.
         The buss number determined by the MIDI engine, if available.
         In ALSA, these two are the client:bus (e.g. 14:0) numbers.
         The bus name is obtained by \texttt{ports::get\_bus\_name()}.
      \item \textsl{Port Number, Port Name, and Port Alias}.
         Not sure what this is, must investigate and fix \textsl{Seq66}
         first.
         The name is accessed by \texttt{get\_port\_name()} and the
         alias is accessed by \texttt{get\_port\_alias()} (available
         only with JACK).
      \item \textsl{Port Type, Port Kind}.
         The type indicates if the bus is for input or output.
         The kind indicates if it is a system, virtual, or normal port.
      \item \textsl{Clock type}.
         Indicates both the port status (unavailable, disabled) and the
         sending of MIDI Clock and Start.
      \item \textsl{Mutex}.
         A recursive locking mutex can be used in processing.
      \item \textsl{MIDI Processing Functions}.
         Provides function to activate/deactivate, start, send system exclusive,
         controlling MIDI Clock.
   \end{itemize}

   Quandary: we need to access the MIDI APIs. The current idea is to use
   these virtual functions to call the virtual function overrides of bus\_in
   and bus\_out. But these functions are not written, and they require
   another bunch of boilerplate code to call the corresponding rtmidi\_in
   and rtmidi\_out functions.

   Alternative: ???



\subsection{Rtl Audio}
\label{subsec:rtl_audio}

   They reside in the \texttt{audio} namespace.
   TBD.

%-------------------------------------------------------------------------------
% vim: ts=3 sw=3 et ft=tex
%-------------------------------------------------------------------------------
